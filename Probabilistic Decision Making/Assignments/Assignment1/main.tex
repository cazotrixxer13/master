\documentclass[a4paper,12pt]{article}
\usepackage{amsmath,amssymb,amsfonts,amsthm,mathtools}
\usepackage{enumitem}
\usepackage{hyperref}
\usepackage{tabularx}
\usepackage{cancel}
\usepackage[margin=0.7in]{geometry}
\usepackage[parfill]{parskip}
\usepackage[utf8]{inputenc}
\usepackage{hyperref}
% graphics
\usepackage{tikz}
\usepackage{pythonhighlight}
\usepackage{graphicx}
\usepackage{subcaption}
% Related to math
\usepackage{longtable}
%\usepackage[ruled,vlined,noline]{algorithm2e}
\usepackage{cleveref}
\usepackage{xcolor}
\usepackage{algorithm}
\usepackage{algpseudocode}
\definecolor{myPurple}{rgb}{.8,0,1}
\newcommand{\R}{\mathbb{R}}
\renewcommand{\vec}[1]{\textbf{#1}}
\newcommand{\mat}[1]{\textbf{\MakeUppercase{#1}}}
\newcommand{\norm}[1]{\Vert#1\Vert}
\setlength\LTleft{0pt}
\setlength\LTright{0pt}
\geometry{margin=2.5cm}


\begin{document}

\begin{titlepage}
    \begin{center}
        \begin{huge}
        %% Update assignment number here
            \textbf{Assignment 1}
        \end{huge}
    \end{center}

    \begin{center}
        \begin{large}
            Probabilistic Decision Making VU, WS 2025/26
        \end{large}
    \end{center}

    \begin{center}
        \begin{tabularx}{\textwidth}{|>{\hsize=.33\hsize}X|>{\hsize=.33\hsize}X|>{\hsize=.33\hsize}X|} 
            \hline
            \multicolumn{3}{|c|}{\textbf{Team Members}} \\
            \hline
            Last name & First name & Matriculation Number \\
            \hline
            Enis & Jusic & 12004830 \\
            \hline
            Lal & Vinayak & 12333144 \\
            \hline

        \end{tabularx}
    \end{center}
\end{titlepage}

\newpage
\section*{Task 1--Probability Spaces [5 Points]}

\subsection*{Task 1.1 [1.5 Points]}
For each of the following candidates $(\Omega, \mathcal{F})$, decide whether $\mathcal{F}$ is a $\sigma$-algebra on the set $\Omega$. Justify your answer in each case. $P(A)$ denotes the powerset of $A$, i.e., the set of all subsets of A.

\begin{enumerate}[label=(\alph*)]
    \item $\Omega = \{1,2,3,4,5,6\}$, $\mathcal{F} = \{\emptyset, \Omega, \{1,2\}, \{3,4\}, \{5,6\}, \{1,2,3,4\}, \{1,2,5,6\}, \{3,4,5,6\}\}$.
    \item $\Omega = \mathbb{R}$, $\mathcal{F} = P(\Omega)$.
    \item $\Omega = \mathbb{N}$, $\mathcal{F} = \{A \subseteq \Omega : A \text{ finite or } \Omega \setminus A \text{ finite}\}$.
    \item $\Omega = \mathbb{N}$, $\mathcal{F} = \{A \subseteq \Omega : A \text{ countable or } \Omega \setminus A \text{ countable}\}$.
\end{enumerate}


\section*{1.1}

For this task, we are using the definition (from the slides) of a $\sigma$-algebra, where $\mathcal{F}$ is called a $\sigma$-algebra over $\Omega$ if:

\begin{enumerate}
    \item $\Omega \in \mathcal{F}$
    \item $\mathcal{F}$ is closed under complement: if $A \in \mathcal{F}$, then also $A^c = \Omega \setminus A \in \mathcal{F}$
    \item $\mathcal{F}$ is closed under countable unions: if $A_1, A_2, \ldots \in \mathcal{F}$, then also $\bigcup_{i=1}^\infty A_i \in \mathcal{F}$
\end{enumerate}

It also follows that:
\begin{itemize}
    \item $\emptyset \in \mathcal{F}$
    \item $\mathcal{F}$ is closed under countable intersections: if $A_1, A_2, \ldots \in \mathcal{F}$, then also $\bigcap_{i=1}^\infty A_i \in \mathcal{F}$
\end{itemize}

\subsection*{(a)}
$\Omega = \{1,2,3,4,5,6\}$, $\mathcal{F} = \{\emptyset, \Omega, \{1,2\}, \{3,4\}, \{5,6\}, \{1,2,3,4\}, \{1,2,5,6\}, \{3,4,5,6\}\}$\\\\
\textbf{Yes, this is a $\sigma$-algebra.}\\\\
The complement of each set is also listed in $\mathcal{F}$:

\[
\begin{aligned}
\emptyset &\leftrightarrow \Omega \\
\{1,2\} &\leftrightarrow \{3,4,5,6\} \\
\{3,4\} &\leftrightarrow \{1,2,5,6\} \\
\{5,6\} &\leftrightarrow \{1,2,3,4\}
\end{aligned}
\]
Also, $\mathcal{F}$ is closed under countable unions:

\[
\begin{aligned}
\{1,2\} \cup \{3,4\} &= \{1,2,3,4\} \\
\{1,2\} \cup \{5,6\} &= \{1,2,5,6\} \\
\{3,4\} \cup \{5,6\} &= \{3,4,5,6\} \\
\{1,2\} \cup \{3,4\} \cup \{5,6\} &= \Omega\\
\{1,2,3,4\} \cup \{5,6\} &= \Omega \\
\{1,2,5,6\} \cup \{3,4\} &= \Omega \\
\{3,4,5,6\} \cup \{1,2\} &= \Omega
\end{aligned}
\]
Any set $\cup$ $\Omega = \Omega$. \\
Any set $\cup$ $\emptyset = $ the set itself. \\
Thus, $\mathcal{F}$ satisfies all properties of a $\sigma$-algebra. 

\subsection*{(b) $\Omega = \mathbb{R}$, $\mathcal{F} = P(\Omega)$}
\textbf{Yes, this is a $\sigma$-algebra.}\\\\
The power set of any set is always a $\sigma$-algebra.
This is because:

\begin{enumerate}
    \item By definition of the powerset, $\Omega = \mathbb{R}$ is a subset of itself, and so $\mathbb{R} \in \mathcal{P}(\mathbb{R})$.
    
    \item Let $A \in \mathcal{F} = \mathcal{P}(\mathbb{R})$. Then $A \subseteq \mathbb{R}$, and that means that the complement
    $\mathbb{R} \setminus A$ is also a subset of $\mathbb{R}$. Therefore $\mathbb{R} \setminus A \in \mathcal{P}(\mathbb{R}) = \mathcal{F}$.
    
    \item Let $A_i$ be a countable family with $A_i \in \mathcal{F}$ for all $i$.
    Then each $A_i \subseteq \mathbb{R}$, and so the union $\bigcup_{i=1}^{\infty} A_i$ is again a subset of $\mathbb{R}$.
    Hence,
    \[
        \bigcup_{i=1}^{\infty} A_i \in \mathcal{P}(\mathbb{R}) = \mathcal{F}.
    \]
\end{enumerate}

\subsection*{(c) $\Omega = \mathbb{N}$, $\mathcal{F} = \{A \subseteq \Omega : A \text{ finite or } \Omega \setminus A \text{ finite}\}$}
\textbf{No, this is not a $\sigma$-algebra.}\\\\
For the complements, we have two options:
\begin{enumerate}
    \item If $A$ is finite, the complement is cofinite, still $\in \mathcal{F}$.
    \item If $A$ is cofinite, the complement is finite, still $\in \mathcal{F}$.
\end{enumerate}
But $\mathcal{F}$ is not closed under \underline{countable} unions.\\
Let $A_k$ be finite sets of $\mathbb{N}$ with one element each, being odd numbers:
\[
A_1 = \{1\}, A_2 = \{3\}, A_3 = \{5\}, \dots
\]
If we take their countable union,
\[
\bigcup_{k=1}^{\infty} A_k = \{1,3,5,\dots\},
\]
this set becomes infinite, and also the complement of $A_k$ the even numbers is infinite too.  
This breaks the definition of a $\sigma$-algebra even though it is closed under complements it is not closed under countable unions. 

\subsection*{(d) $\Omega = \mathbb{N}$, $\mathcal{F} = \{A \subseteq \Omega : A \text{ countable or } \Omega \setminus A \text{ countable}\}$}
\textbf{Yes, this is a $\sigma$-algebra.}\\\\
Every subset of $\Omega = \mathbb{N}$ is countable since $\mathbb{N}$ itself is countable by definition. By taking the same argument as in we defined in (b) that  $\mathcal{F} = \mathcal{P}(\Omega)$, which is always a $\sigma$-algebra, we conclude $\mathcal{F}$ to be a $\sigma$-algebra.

\begin{enumerate}
    \item It contains $\Omega$, so $\Omega \in \mathcal{P}(\Omega)$.
    \item If $A \in \mathcal{P}(\Omega)$, then its complement $\Omega \setminus A$ is also a subset of $\Omega$, 
    so $\Omega \setminus A \in \mathcal{P}(\Omega)$.
    \item Any countable union of subsets of $\Omega$ is again a subset of $\Omega$, 
    thus $\bigcup_{i=1}^{\infty} A_i \in \mathcal{P}(\Omega)$.
\end{enumerate}
The countable union of countable sets is again countable and therefore $\mathcal{F} = \mathcal{P}(\Omega)$ is a $\sigma$-algebra. 

\subsection*{Task 1.2 [1.5 Points]}
For each of the following candidates $(\Omega, \mathcal{F})$, decide whether $\mathcal{F}$ is a $\sigma$-algebra on the set $\Omega$. If you think $\mathcal{F}$ is not a $\sigma$-algebra on $\Omega$, provide the smallest $\sigma$-algebra on $\Omega$ that contains all sets $\in$ $\mathcal{F}$.

\begin{enumerate}[label=(\alph*)]
    \item $\Omega = \{1,2,3,4,5,6\}$, $\mathcal{F} = \{\emptyset, \Omega, \{1,2\}, \{3,4,5,6\}, \{1,3\}, \{2,4,5,6\}\}$.
    \item $\Omega = \{1,2,3,4,5,6\}$, $\mathcal{F} = \{\emptyset, \Omega, \{2,4,6\}, \{1,3,5\}, \{1,2,3\}, \{4,5,6\}\}$.
    \item $\Omega = \{1,2,3,4,5\}$, $\mathcal{F} = \{\emptyset, \Omega, \{1,2\}, \{3,4,5\}, \{1,2,3,4\}, \{5\}, \{1,2,5\}\}$.
\end{enumerate}

\section*{1.2}
We again use the definition above from the slide, closure under complements, closure under countable unions.

\subsection*{(a) $\Omega=\{1,2,3,4,5,6\}$, $\mathcal F=\{\emptyset,\Omega,\{1,2\},\{3,4,5,6\},\{1,3\},\{2,4,5,6\}\}$}
\textbf{No, this is not a $\sigma$-algebra.}\\\\
Basically the unions of the sets are missing which again introduce new complement sets.
Smallest $\sigma$-algebra:
\[
\begin{aligned}
&\Big\{\emptyset,\ \Omega,\ \{1\},\ \{2\},\ \{3\},\ \{4,5,6\},\ \{1,2\},\ \{1,3\},\ \{1,4,5,6\}, \{2,3\},\ \\ 
&\{2,4,5,6\},\ \{3,4,5,6\},\ \{1,2,3\},\ \{1,2,4,5,6\},\ \{1,3,4,5,6\},\ \{2,3,4,5,6\}\Big\}.
\end{aligned}
\]
For each set we have the complement and F is now closed under countable unions.

\subsection*{(b) $\Omega=\{1,2,3,4,5,6\}$, $\mathcal F=\{\emptyset,\Omega,\{2,4,6\},\{1,3,5\},\{1,2,3\},\{4,5,6\}\}$}
\textbf{No, this is not a $\sigma$-algebra.}\\\\
Again the unions of the sets are missing which introduce new complement sets.
Smallest $\sigma$-algebra:
\[
\begin{aligned}
&\Big\{\emptyset,\ \Omega, \{2,4,6\}, \{1,3,5\}, \{1,2,3\}, \{4,5,6\}, \{1,2,3,4,6\}, \{5\}, \\&\{2,4,5,6\}, \{1,3\}, \{1,2,3,5\}, \{4,6\}, \{1,3,4,6\}, \{2,5\}, \{2\}, \{1,3,4,5,6\}\Big\}.
\end{aligned}
\]
For each set we have the complement and F is now closed under countable unions.

\subsection*{(c) $\Omega=\{1,2,3,4,5\}$, $\mathcal F=\{\emptyset,\Omega,\{1,2\},\{3,4,5\},\{1,2,3,4\},\{5\},\{1,2,5\}\}$}
\textbf{No, this is not a $\sigma$-algebra.}\\\\
Here only one complement $\{3,4\}$ of $\{1,2,5\}$ is missing.\\
So the smalles $\sigma$-algebra:
\[
\Big\{\emptyset,\Omega,\{1,2\},\{3,4,5\},\{1,2,3,4\},\{5\},\{1,2,5\},\{3,4\}\Big\}.
\]
For each set we now have the complement and F is closed under countable unions.

\subsection*{Task 1.3 [1 Point]}
For each of the following $(\Omega, \mathcal{F}, \mathbb{P})$, decide whether it is a valid probability triple. Justify your answer in each case.

\begin{enumerate}[label=(\alph*)]
    \item $\Omega = \{1,2,3,4,5\}, \quad \mathcal{F} = P(\Omega), \quad \mathbb{P}(A) = \sum_{n \in A} p_n,$ with $(p_1,\dots,p_5) = (0.2,0.3,0.4,0.1,0)$.
    \item $\Omega = \mathbb{N} \setminus \{0\}, \quad \mathcal{F} = P(\Omega), \quad \mathbb{P}(A) = \sum_{n \in A} \frac{1}{2^n}$.
    \item $\Omega = \mathbb{N} \setminus \{0\}, \quad \mathcal{F} = P(\Omega), \quad \mathbb{P}(A) = \sum_{n \in A} \frac{1}{n^2}$.
    \item $\Omega = [0,1], \quad \mathcal{F} = P(\Omega), \quad \mathbb{P}(A) = 0.6\,\delta_0(A) + 0.4\,\delta_1(A)$ \quad with $\delta_x(A) :=
    \begin{cases}
        1 & \text{if } x \in A,\\
        0 & \text{if } x \notin A.
    \end{cases}$

\end{enumerate}


\newpage

\section*{1.3}
For each candidate $(\Omega,\mathcal F,\mathbb P)$ we will decide if it is a probability triple which is again defined in the slides as follows:

\begin{itemize}
    \item Let $\Omega$ be a non-empty set (sample space),
    \item $\mathcal{F}$ be a sigma-algebra over $\Omega$
    \item $\mathbb{P}: \mathcal{F} \mapsto [0,1]$ be a function (probability measure) with
    \begin{itemize}
        \item $\mathbb{P}(\Omega) = 1$
        \item For any disjoint $A_1, A_2, A_3, \ldots$ from $\mathcal{F}$,
        \[
            \mathbb{P}\!\left( \bigcup_{i=1}^{\infty} A_i \right)
            = \sum_{i=1}^{\infty} \mathbb{P}(A_i)
            \quad \text{(sigma-additivity)}
        \]
    \end{itemize}
    (Note that this includes finite unions as well, by setting all but finitely many $A_i = \emptyset$.)
\end{itemize}
Any such triple $(\Omega, \mathcal{F}, \mathbb{P})$ is called a probability space.  

\subsection*{(a) $\Omega=\{1,2,3,4,5\}, \mathcal F=\mathcal P(\Omega),
\mathbb P(A)=\sum_{n\in A}p_n,\ (p_1,\dots,p_5)=(0.2,0.3,0.4,0.1,0)$}
\textbf{Yes, this is a probability space.}\\\\
It holds that $0 \le p_n \le 1$ and 
\[
\sum_{n=1}^5 p_n=0.2+0.3+0.4+0.1+0=1
\]
and so $\mathbb P(\Omega)=1$.

\subsection*{(b) $\Omega=\mathbb N\setminus\{0\},\quad \mathcal F=\mathcal P(\Omega),\quad
\mathbb P(A)=\sum_{n\in A}2^{-n}$}
\textbf{Yes, this is a probability space.}\\\\
This is a geometric series with 
\[
\sum_{n=1}^\infty \frac{1}{2^n}=\frac{1}{2}+\frac{1}{4}+\frac{1}{8}+...=1,
\]
so $\mathbb P(\Omega)=1$ and also $0\le \mathbb P(A)\le 1$ for all $A$.

\subsection*{(c) $\Omega=\mathbb N\setminus\{0\}, \mathcal F=\mathcal P(\Omega),
\mathbb P(A)=\sum_{n\in A}n^{-2}$}
\textbf{No, this is not a probability space.}\\\\
We have again a known series called the Basel Problem
\[
\mathbb P(\Omega)=\sum_{n=1}^\infty \frac{1}{n^2}=\frac{1}{1^2}+\frac{1}{2^2}+\frac{1}{3^2}+...=\frac{\pi^2}{6}>1, 
\]
so $\mathbb P(\Omega)\neq 1$.

\subsection*{(d)  $\Omega=[0,1],\quad \mathcal F=\mathcal P(\Omega),\quad
\mathbb P(A)=0.6\,\delta_0(A)+0.4\,\delta_1(A)$}
\textbf{Yes, this is a probability space.}\\\\
Clearly $0\le\mathbb P(A)\le 1$ and $\mathbb P(\Omega)=0.6+0.4=1$.
If $A_i$ are disjoint, at most one $A_i$ contains $0$ and at most one contains $1$, where
\begin{align*}
\mathbb P\!\left(\bigcup_i^2 A_i\right)
&=0.6\,\delta_0\!\left(\bigcup_i^2 A_i\right)+0.4\,\delta_1\!\left(\bigcup_i^2 A_i\right)\\
&=\sum_i \big(0.6\,\delta_0(A_i)+0.4\,\delta_1(A_i)\big)\\
&=0.6\times1+0.4\times1\\
&=1
\end{align*}
so $\mathbb P(\Omega)=1$.


\section*{Task 1.4 [1 Points]}
Let $(\Omega, \mathcal{F}, \mathbb{P})$ and $(\Omega, \mathcal{F}, \mathbb{Q})$ be two probability triples, i.e., 
$\Omega$ is a non-empty sample space, $\mathcal{F}$ is a $\sigma$-algebra on $\Omega$, 
and $\mathbb{P}, \mathbb{Q} : \mathcal{F} \to [0,1]$ are probability measures.

\begin{enumerate}[label=(\alph*)]

\item Assume that
\[
    \mathbb{P}(A) = \mathbb{Q}(A) \quad \forall A \in \mathcal{F} \text{ where } \mathbb{P}(A) \le \frac{1}{2}.
\]
Prove the following statement:
\[
    \mathbb{P}(A) = \mathbb{Q}(A) \quad \forall A \in \mathcal{F}.
\]

\item Assume now instead that
\[
    \mathbb{P}(A) = \mathbb{Q}(A) \quad \forall A \in \mathcal{F} \text{ where } \mathbb{P}(A) < \frac{1}{2}.
\]
Do we still have $\mathbb{P}(A) = \mathbb{Q}(A)$ for all $A \in \mathcal{F}$? 
If you think so, prove this. 
If you believe the opposite, provide a counterexample, 
i.e., a concrete example of $\Omega, \mathcal{F}, \mathbb{P}$ and $\mathbb{Q}$ that do not satisfy this.
\end{enumerate}

\subsection*{1.4}
Let $(\Omega,\mathcal F,\mathbb P)$ and $(\Omega,\mathcal F,\mathbb Q)$ be probability spaces.

\subsubsection*{(a)}

We know that the probability of the set A $\mathbb P(A)$ + its complement $\mathbb P(A^c)$ together must equal to $\mathbb P(\Omega)=1$ by the definition of probability spaces.
\[
\mathbb P(A)+\mathbb P(A^c)=\mathbb P(\Omega)=1.
\]
So it has to hold that either $\mathbb P(A)\le \tfrac12$ or $\mathbb P(A^c)\le \tfrac12$.\\
If $\mathbb P(A)\le\tfrac12$, the assumption already gives us $\mathbb Q(A)=\mathbb P(A)$.\\
But if the complement $\mathbb P(A^c)\le\tfrac12$, we also have from the assumption that $\mathbb P(A^c)=\mathbb Q(A^c)$ holds. That means that we can define $\mathbb Q(A)$ as $1-\mathbb Q(A^c)$ 
So yes the statement is correct.

\subsubsection*{(b)} 
The statement can fail with the strict inequality now simply because we exclude $\mathbb P(A)=\frac12$. We can easily construct a counterexample and show that it not holds.\\\\
Let $\Omega=\{1,2\}$, $\mathcal F=\mathcal P(\Omega)$.\\
We define $\mathbb P(\{1\})=\mathbb P(\{2\})=\tfrac12$ where obviously $\mathbb P(\emptyset)=0$, $\mathbb P(\Omega)=1$ holds automatically to be a proper probability measure.\\
On the counterpart we now define $\mathbb Q(\{1\})=\tfrac23$, $\mathbb Q(\{2\})=\tfrac13$, and again $\mathbb Q(\emptyset)=0$, $\mathbb Q(\Omega)=1$. Then both $\mathbb P$ and $\mathbb Q$ are probability measures.

Now the only set with $\mathbb P(A)<\tfrac12$ is $A=\emptyset$. And since we defined $\mathbb P(\emptyset)=0$ but also $\mathbb Q(\emptyset)=0$ it shows that
\[
\mathbb P(A)=\mathbb Q(A)\quad\text{for all }A\text{ with }\mathbb P(A)<\tfrac12
\]
is satisfied. But actually we have defined that $\mathbb P(\{1\})=\tfrac12$ and also $\mathbb Q(\{1\})=\tfrac23$ where clearly $\tfrac12 \neq \tfrac23$ so $\mathbb P\ne\mathbb Q$ on $\mathcal F$.
It does not hold when the assumption uses $<$ instead of $\le$.

\newpage

%%%%%%%%%%%%%%%%%%%%%%%%%%%%%----- 1 ------%%%%%%%%%%%%%%%%%%%%%%%%%%%
\section*{\centering Task 2 : Sigma-Algebras}

\hrule

%%%%%%%%%%%%%%%%%%%%%%%%%%%%%----- 2.1 ------%%%%%%%%%%%%%%%%%%%%%%%%%%%
\vspace{5mm}

\subsection*{Task 2.1}
\vspace{2mm}
\begin{normalsize}
    A set of sets $\mathit{E}$ is a sigma-algebra over a finite set $\Omega$ iff $\mathit{E}$ meets these conditions,
    \begin{enumerate}
        \item $\Phi , \Omega \in \mathit{E}$
        \item If a set $\mathit{A} \in \mathit{E}$ then $\mathit{A}^{\mathit{c}} \in \mathit{E}$
        \item if sets $\mathit{A}_{1}, \mathit{A}_{2}, . . . \in \mathit{E}$ then also $\bigcup_{i=1}^{\infty} \mathit{A}_{i} \in \mathit{E}$
    \end{enumerate}
\end{normalsize}

\vspace{2mm}

\begin{algorithm}
\caption{To check of the given set of sets $\mathit{E}$ is a $\sigma-$algebra over a finite set $\Omega$}
\begin{algorithmic}[1]
\Require Sample space $\Omega$, collection of sets $\mathcal{E}$
\State Let $\varnothing \gets \Omega - \Omega$
\If{$\Omega \notin \mathcal{E}$} \State \Return \textbf{false} \EndIf
\If{$\varnothing \notin \mathcal{E}$} \State \Return \textbf{false} \EndIf

\ForAll{$A \in \mathcal{E}$}
    \If{$\Omega - A \notin \mathcal{E}$}
        \State \Return \textbf{false}
    \EndIf
\EndFor

\State $U \gets \varnothing$
\ForAll{$A \in \mathcal{E}$}
    \State $U \gets U \cup A$
\EndFor
\If{$U \notin \mathcal{E}$} \State \Return \textbf{false} \EndIf

\State \Return \textbf{true}
\end{algorithmic}
\end{algorithm}

\begin{normalsize}
    The time the implemented algorithm takes is $\mathbf{O(|E|^2)}$
\end{normalsize}

\vspace{2mm}
\begin{normalsize}
    \textbf{Test Cases:}
\end{normalsize}

{\renewcommand{\arraystretch}{1.05}\large
\begin{longtable}{p{0.23\textwidth} p{0.50\textwidth} p{0.20\textwidth}}
\hline
\textbf{$\Omega$} & \textbf{$\mathit{E}$} & \textbf{Is Sigma Algebra?} \\
\hline
\endfirsthead

\hline
\endfoot

$\{1,2,3,4,5,6\}$ &
\(\varnothing\), \(\Omega\) &
True \\

$\{1,2,3,4,5,6\}$ &
\(\varnothing\), \(\{1,3,5\}\), \(\{2,4,6\}\), \(\Omega\) &
True \\

$\{1,2,3,4,5,6\}$ &
\(\varnothing\), \(\{1,2\}\), \(\{3,4\}\), \(\{5,6\}\), \(\{3,4,5,6\}\), \(\{1,2,5,6\}\), \(\{1,2,3,4\}\), \(\Omega\) &
True \\

$\{1,2,3,4,5,6\}$ &
\(\varnothing\), \(\{1\}\), \(\{2,3,4,6\}\), \(\Omega\) &
False \\

$\{1,2,3,4,5,6\}$ &
\(\mathcal{P}(\Omega)\) &
True \\

\end{longtable}
}
%%%%%%%%%%%%%%%%%%%%%%%%%%%%%----- 2.1 ------%%%%%%%%%%%%%%%%%%%%%%%%%%%
\pagebreak
%%%%%%%%%%%%%%%%%%%%%%%%%%%%%----- 2.2 ------%%%%%%%%%%%%%%%%%%%%%%%%%%%
\subsection*{Task 2.2}
\vspace{2mm}

\begin{algorithm}[H]
\caption{Construction of the Smallest Sigma-Algebra Containing a Given Collection of Sets}
\begin{algorithmic}[1]
\Require Sample space $\Omega$, collection of sets $\mathcal{E}$
\Ensure Smallest sigma-algebra on $\Omega$ containing all sets in $\mathcal{E}$ that are subsets of $\Omega$
\State Initialize $\text{smallest\_sigma\_algebra} \gets [\ ]$
\State Initialize $\text{valid\_sets} \gets [\ ]$
\State Let $\varnothing \gets \Omega - \Omega$
\For{each $A$ in $\mathcal{E}$}
    \If{$A \subseteq \Omega$}
        \State Append $A$ to $\text{valid\_sets}$
    \EndIf
\EndFor
\State Append all elements of $\text{valid\_sets}$ to $\text{smallest\_sigma\_algebra}$
\If{$\Omega \notin \text{valid\_sets}$}
    \State Append $\Omega$ to $\text{smallest\_sigma\_algebra}$
\EndIf
\If{$\varnothing \notin \text{valid\_sets}$}
    \State Append $\varnothing$ to $\text{smallest\_sigma\_algebra}$
\EndIf
\For{each $A$ in $\text{valid\_sets}$}
    \State Let $A^{c} \gets \Omega - A$
    \If{$A^{c} \notin \text{smallest\_sigma\_algebra}$}
        \State Append $A^{c}$ to $\text{smallest\_sigma\_algebra}$
    \EndIf
\EndFor
\State \Return $\text{smallest\_sigma\_algebra}$
\end{algorithmic}
\end{algorithm}

\vspace{2mm}
\begin{normalsize}
    \textbf{Test Cases:}
\end{normalsize}

{\renewcommand{\arraystretch}{1.05}\large
\begin{longtable}{p{0.23\textwidth} p{0.25\textwidth} p{0.45\textwidth}}
\hline
\textbf{$\Omega$} & \textbf{$\mathit{E}$} & \textbf{Smallest Sigma Algebra} \\
\hline
\endfirsthead

\hline
\endfoot

$\{1,2,3,4,5,6\}$ &
\(\{1,2\}\), \(\{8,4\}\) &
\(\varnothing\), \(\{1,2\}\), \(\{3,4,5,6\}\), \(\Omega\) \\

$\{1,2,3,4,5,6\}$ &
\(\{1\}\) &
\(\varnothing\), \(\{1\}\), \(\{2,3,4,5,6\}\), \(\Omega\) \\

$\{1,2,3,4,5,6\}$ &
\(\{3,4,5\}\) &
\(\varnothing\), \(\{3,4,5\}\), \(\{1,2,6\}\), \(\Omega\) \\

$\{1,2,3,4,5,6\}$ &
\(\varnothing\) &
\(\varnothing\), \(\Omega\) \\

$\{1,2,3,4,5,6\}$ &
\(\Omega\) &
\(\varnothing\), \(\Omega\) \\

\end{longtable}
}

%%%%%%%%%%%%%%%%%%%%%%%%%%%%%----- 2.2 ------%%%%%%%%%%%%%%%%%%%%%%%%%%%
\pagebreak
%%%%%%%%%%%%%%%%%%%%%%%%%%%%%----- 2.3 ------%%%%%%%%%%%%%%%%%%%%%%%%%%%
\subsection*{Task 2.3}
\vspace{2mm}

\begin{large}
    \textbf{If $\mathit{F}_{1}$ and $\mathit{F}_{2}$ are two sigma algebras over $\Omega$}

    \textbf{To Check if $\mathit{F}_{1} \cap \mathit{F}_{2}$ is a sigma algebra over $\Omega$}
\end{large}

\vspace{2mm}

\begin{large}
\begin{enumerate}
    \item Since, $\Omega \in \mathit{F}_{1}$ and $\Omega \in \mathit{F}_{2}$, therefore, $\Omega \in \mathit{F}_{1} \cap \mathit{F}_{2}$ holds True.
    
    \item If $\mathit{A} \in \mathit{F}_{1}$ and $\mathit{A} \in \mathit{F}_{2}$, then $\mathit{A} \in \mathit{F}_{1} \cap \mathit{F}_{2}$. Since $\mathit{F}_{1}$ and $\mathit{F}_{2}$ are sigma-algebras, $\mathit{A}^{c} \in \mathit{F}_{1}$ and $\mathit{A}^{c} \in \mathit{F}_{2}$ and hence $\mathit{A}^{c} \in \mathit{F}_{1} \cap \mathit{F}_{2}$
    
    \item If $A_1, A_2, . . . \in \mathit{F}_{1} \cap \mathit{F}_{2}$ then $A_1, A_2, . . . \in \mathit{F}_{1}$ and $A_1, A_2, . . . \in \mathit{F}_{2}$.
    
    Since $\mathit{F_1}$ and $\mathit{F_2}$ are sigma-algebras, $\bigcup_{i=1}^{\infty} \mathit{A}_{i} \in \mathit{F_1}$ and $\bigcup_{i=1}^{\infty} \mathit{A}_{i} \in \mathit{F_2}$.

    Hence $\bigcup_{i=1}^{\infty} \mathit{A}_{i} \in \mathit{F}_{1} \cap \mathit{F}_{2}$ holds True.
\end{enumerate}
\vspace{2mm}
    Since all the conditions meet, $\mathit{F}_{1} \cap \mathit{F}_{2}$ is a sigma algebra over $\Omega$
\end{large}
%%%%%%%%%%%%%%%%%%%%%%%%%%%%%----- 2.3 ------%%%%%%%%%%%%%%%%%%%%%%%%%%%
\vspace{6mm}
%%%%%%%%%%%%%%%%%%%%%%%%%%%%%----- 2.4 ------%%%%%%%%%%%%%%%%%%%%%%%%%%%
\subsection*{Task 2.4}
\vspace{2mm}

\begin{large}
    
    \textbf{If $\mathit{F}_{1}$ and $\mathit{F}_{2}$ are two sigma algebras over $\Omega_1$ and $\Omega_2$}

    \textbf{To Check if $\mathit{F}_{1} \cup \mathit{F}_{2}$ is a sigma algebra over $\Omega_1 \cup \Omega_2$}
\end{large}

\vspace{2mm}

\begin{large}
    Since $\Omega_1 \in \mathit{F}_{1}$ and $\Omega_2 \in \mathit{F}_{2}$, we assume $\Omega_1 \cup  \Omega_2 \in \mathit{F}_{1} \cup \mathit{F}_{2}$, but this is not True.

    Let us assume $\Omega_1 = \{1\}$ and $\Omega_2 = \{2\}$ 

    Then $\mathit{F}_{1} = \{ \Phi , \{1\} \}$ and $\mathit{F}_{2} = \{ \Phi , \{2\} \}$
    
    $\mathit{F}_{1} \cup \mathit{F}_{2} = \{ \Phi , \{1\}, \{2\} \}$ and does not contain $\Omega_1 \cup \Omega_2 = \{1, 2\}$

    Therefore, $\Omega_1 \cup  \Omega_2 \notin \mathit{F}_{1} \cup \mathit{F}_{2}$

    \vspace{2mm}

    Hence we can say, $\mathit{F}_{1} \cup \mathit{F}_{2}$ is not a sigma algebra over $\Omega_1 \cup \Omega_2$
\end{large}
%%%%%%%%%%%%%%%%%%%%%%%%%%%%%----- 2.4 ------%%%%%%%%%%%%%%%%%%%%%%%%%%%
%%%%%%%%%%%%%%%%%%%%%%%%%%%%%%%%%%%%%%%%%%%%%%%%%%%%%%%%%%%%%%%%%%%%%%



















\newpage
\section*{Task 3 – Distribution Functions [5 Points]}

\subsection*{Task 3.1 [3 points]}
Let
\[
\mathcal{N}(x; \mu, \sigma^2) := 
\frac{1}{\sqrt{2\pi\sigma^2}}
\exp\!\left(
-\frac{(x - \mu)^2}{2\sigma^2}
\right)
\]
denote the univariate \emph{Gaussian} probability density function (pdf) with mean $\mu \in \mathbb{R}$ and variance $\sigma^2 > 0$.

Show that the product of two Gaussian pdfs is again an (unnormalized) Gaussian, i.e.,
\[
\mathcal{N}(x; a, \sigma_a^2)\,\mathcal{N}(x; b, \sigma_b^2)
= Z \cdot \mathcal{N}(x; \mu_{ab}, \sigma_{ab}^2)
\quad\text{with}\quad
\mu_{ab} = \frac{\sigma_b^2 a + \sigma_a^2 b}{\sigma_a^2 + \sigma_b^2}
\quad\text{and}\quad
\sigma_{ab}^2 = \frac{\sigma_a^2\sigma_b^2}{\sigma_a^2 + \sigma_b^2},
\]
where $Z = \int \mathcal{N}(x; a, \sigma_a^2)\,\mathcal{N}(x; b, \sigma_b^2)\,dx$ is the \emph{partition function}, which can also be written as evaluating a different Gaussian at the point $a$: $Z = \mathcal{N}(a; b, \sigma_a^2 + \sigma_b^2).$ Clearly show all steps in your derivation. Your final result should exactly recover the equation $\mathcal{N}(x; a, \sigma_a^2)\,\mathcal{N}(x; b, \sigma_b^2) = \mathcal{N}(a; b, \sigma_a^2 + \sigma_b^2)\,\mathcal{N}(x; \mu_{ab}, \sigma_{ab}^2)$
with $\mu_{ab}, \sigma_{ab}^2$ as above.


\paragraph{Hints:}
\begin{itemize}
    \item Write down the product using the definition above and make use of the rules of the \(\exp\) function to
    transform the product into a single exponential form.
    \item After some algebraic manipulation, transform the exponent such that the term \(x^2\) has no leading
    coefficient in the numerator, which should allow you to read off \(\mu_{ab}\) and \(\sigma_{ab}^2\).
    \item Complete the square in the exponent. The quantity that is added to the resulting quadratic will be
    related to \(N(a; b, \sigma_a^2 + \sigma_b^2)\).
    \item \textbf{Remark:} Here, we are multiplying two Gaussian density functions (pointwise). This is not the same
    as multiplying the corresponding Gaussian random variables.
\end{itemize}

\subsection*{3.1}

\begin{align*}
\mathcal N(x;a,\sigma_a^2)\,\mathcal N(x;b,\sigma_b^2)
&=
\left(
  \frac{1}{\sqrt{2\pi\sigma_a^2}}
  \exp\!\left( -\frac{(x-a)^2}{2\sigma_a^2} \right)
\right)
\left(
  \frac{1}{\sqrt{2\pi\sigma_b^2}}
  \exp\!\left( -\frac{(x-b)^2}{2\sigma_b^2} \right)
\right)\\
&=
\frac{1}{\sqrt{2\pi\sigma_a^2}}\,
\frac{1}{\sqrt{2\pi\sigma_b^2}}\,
\exp\!\left( -\frac{(x-a)^2}{2\sigma_a^2} \right)
\exp\!\left( -\frac{(x-b)^2}{2\sigma_b^2} \right)
\\[0.5em]
&=
\frac{1}{2\pi\sqrt{\sigma_a^2\sigma_b^2}}\,
\exp\!\left(
  -\frac{(x-a)^2}{2\sigma_a^2}
\right)
\exp\!\left(
  -\frac{(x-b)^2}{2\sigma_b^2}
\right)
\\[0.5em]
&\quad\text{(Now using the first hint we know that } e^{-a}e^{-b}=e^{-a-b} \text{), so we get}
\\[0.5em]
&=
\frac{1}{2\pi\sqrt{\sigma_a^2\sigma_b^2}}\,
\exp\!\left(
  -\frac{(x-a)^2}{2\sigma_a^2}
  -\frac{(x-b)^2}{2\sigma_b^2}
\right)
\tag{1}
\end{align*}\\
Now we can expand the squares in the exponent to get the first and second binomial formulas
\begin{align*}
(x-a)^2 &= x^2 - 2ax + a^2, \\
(x-b)^2 &= x^2 - 2bx + b^2.
\tag{2}
\end{align*}
and substitute these into the exponent term of what we got in \((1)\)
\begin{align*}
&-\frac{(x-a)^2}{2\sigma_a^2}
-\frac{(x-b)^2}{2\sigma_b^2}
\\[0.5em]
&= -\frac{1}{2\sigma_a^2}(x^2 - 2ax + a^2)
   -\frac{1}{2\sigma_b^2}(x^2 - 2bx + b^2)
\\[0.5em]
&= -\frac{1}{2\sigma_a^2}x^2 + \frac{\cancel{2}a}{\cancel{2} \sigma_a^2}x - \frac{a^2}{2\sigma_a^2}
   -\frac{1}{2\sigma_b^2}x^2 + \frac{\cancel{2}b}{\cancel{2}\sigma_b^2}x - \frac{b^2}{2\sigma_b^2}
\\[0.5em]
&= -\left(\frac{1}{2\sigma_a^2} + \frac{1}{2\sigma_b^2}\right)x^2
   +\left(\frac{a}{\sigma_a^2} + \frac{b}{\sigma_b^2}\right)x
   -\left(\frac{a^2}{2\sigma_a^2} + \frac{b^2}{2\sigma_b^2}\right).
\tag{3}
\end{align*}
With the second hint we factor out the \(-\frac12\) to make the structure clearer to have $x^2$ with no leading coefficient in the numerator and also negate the +/- signs

\begin{align*}
&=-\frac12\left[
 \left(\frac{1}{\sigma_a^2} + \frac{1}{\sigma_b^2}\right)x^2
 -2\left(\frac{a}{\sigma_a^2} + \frac{b}{\sigma_b^2}\right)x
 +\left(\frac{a^2}{\sigma_a^2} + \frac{b^2}{\sigma_b^2}\right)
\right].
\tag{4}
\end{align*}
We can rewrite the current exponent by factoring out \(\frac{1}{\sigma_{ab}^2}\) from the quadratic terms and introduce the combined variance \(\sigma_{ab}^2\) like

\begin{align*}
\frac{1}{\sigma_{ab}^2}
:= \frac{1}{\sigma_a^2} + \frac{1}{\sigma_b^2}.
\tag{5}
\end{align*}
If we invert both sides we obtain

\[
\sigma_{ab}^2
= \frac{1}{\dfrac{1}{\sigma_a^2} + \dfrac{1}{\sigma_b^2}}
= \frac{1}{\dfrac{\sigma_a^2 + \sigma_b^2}{\sigma_a^2\sigma_b^2}}
= \frac{\sigma_a^2\sigma_b^2}{\sigma_a^2 + \sigma_b^2}.
\tag{6}
\]
Substituting \((5)\) in \((4)\) we get
\begin{align*}
&=
-\frac12\left[
 \frac{1}{\sigma_{ab}^2}x^2
 -2\left(\frac{a}{\sigma_a^2} + \frac{b}{\sigma_b^2}\right)x
 +\left(\frac{a^2}{\sigma_a^2} + \frac{b^2}{\sigma_b^2}\right)
\right]
\\[0.5em]
&=
-\frac{1}{2\sigma_{ab}^2}
\left[
 x^2
 -2\sigma_{ab}^2\left(\frac{a}{\sigma_a^2} + \frac{b}{\sigma_b^2}\right)x
 +\sigma_{ab}^2\left(\frac{a^2}{\sigma_a^2} + \frac{b^2}{\sigma_b^2}\right)
\right].
\tag{7}
\end{align*}
So the whole product is now

\begin{align*}
\mathcal N(x;a,\sigma_a^2)\,\mathcal N(x;b,\sigma_b^2)
&=
\frac{1}{2\pi\sqrt{\sigma_a^2\sigma_b^2}}\,
\exp\!\left\{
-\frac{1}{2\sigma_{ab}^2}
\left[
 x^2
 -2\sigma_{ab}^2\left(\frac{a}{\sigma_a^2} + \frac{b}{\sigma_b^2}\right)x
 +\sigma_{ab}^2\left(\frac{a^2}{\sigma_a^2} + \frac{b^2}{\sigma_b^2}\right)
\right]
\right\}.
\tag{8}
\end{align*}
Now we compare the exponent in (7) with the standard Gaussian exponent form

\[
-\frac{1}{2\sigma_{ab}^2}
\left[
x^2 - 2\mu_{ab}x
\right].
\]
The coefficients of \(x^2\) and \(x\) must match.  
From (7):\\
The coefficient of \(x^2\) is \(1\), which already matches.\\
The coefficient of \(-2x\) is \(\sigma_{ab}^2\left(\frac{a}{\sigma_a^2}+\frac{b}{\sigma_b^2}\right)\).
So we can now alos read off
\[
\mu_{ab}
= \sigma_{ab}^2\left(\frac{a}{\sigma_a^2}+\frac{b}{\sigma_b^2}\right).
\tag{9}
\]
Substitute (6) into (9)

\begin{align*}
\mu_{ab}
&= \frac{\sigma_a^2\sigma_b^2}{\sigma_a^2 + \sigma_b^2}
\left(\frac{a}{\sigma_a^2}+\frac{b}{\sigma_b^2}\right)
\\[0.8em]
&= \frac{\sigma_a^2\sigma_b^2}{\sigma_a^2 + \sigma_b^2}
\left(\frac{a\sigma_b^2 + b\sigma_a^2}{\sigma_a^2\sigma_b^2}\right)
\\[0.8em]
&= \frac{\sigma_b^2a + \sigma_a^2b}{\sigma_a^2 + \sigma_b^2}.
\tag{10}
\end{align*}
Then \((7)\) becomes
\[
-\frac{1}{2\sigma_{ab}^2}
\left[ x^2 - 2\sigma_{ab}^2 \Big(\frac{a}{\sigma_a^2} + \frac{b}{\sigma_b^2}\Big)\,x + \sigma_{ab}^2 \Big(\frac{a^2}{\sigma_a^2} + \frac{b^2}{\sigma_b^2}\Big) \right].
\tag{11}
\]
and the quadratic term inside the brackets is actually just
\[
x^2 - 2\sigma_{ab}^2 \Big(\frac{a}{\sigma_a^2} + \frac{b}{\sigma_b^2}\Big)\,x
=
x^2 - 2\mu_{ab}x.
\tag{12}
\]
With the third hint we can now complete the square for the expression
\(x^2 - 2\mu_{ab}x\)

\begin{align*}
x^2 - 2\mu_{ab}x
&=
x^2 - 2\mu_{ab}x + (\mu_{ab}^2 - \mu_{ab}^2)
\\[0.4em]
&= (x^2 - 2\mu_{ab}x + \mu_{ab}^2) - \mu_{ab}^2
\\[0.4em]
&= (x - \mu_{ab})^2 - \mu_{ab}^2.
\tag{13}
\end{align*}
We substitute this back into \((11)\) and get

\begin{align*}
x^2 - 2\sigma_{ab}^2 \Big(\frac{a}{\sigma_a^2} + \frac{b}{\sigma_b^2}\Big)\,x + \sigma_{ab}^2 \Big(\frac{a^2}{\sigma_a^2} + \frac{b^2}{\sigma_b^2}\Big)
&= (x-\mu_{ab})^2 - \mu_{ab}^2 + \sigma_{ab}^2 \Big(\frac{a^2}{\sigma_a^2} + \frac{b^2}{\sigma_b^2}\Big).
\tag{14}
\end{align*}
and the whole exponent becomes

\begin{align*}
-\frac{1}{2\sigma_{ab}^2}
\left[ x^2 - 2\sigma_{ab}^2 \Big(\frac{a}{\sigma_a^2} + \frac{b}{\sigma_b^2}\Big)\,x + \sigma_{ab}^2 \Big(\frac{a^2}{\sigma_a^2} + \frac{b^2}{\sigma_b^2}\Big) \right]
&=
-\frac{1}{2\sigma_{ab}^2}
\left[ (x-\mu_{ab})^2 - \mu_{ab}^2 + \sigma_{ab}^2 \Big(\frac{a^2}{\sigma_a^2} + \frac{b^2}{\sigma_b^2}\Big) \right]
\\[0.4em]
&=
-\frac{(x-\mu_{ab})^2}{2\sigma_{ab}^2}
+\frac{1}{2\sigma_{ab}^2}\left(\mu_{ab}^2 - \sigma_{ab}^2 \Big(\frac{a^2}{\sigma_a^2} + \frac{b^2}{\sigma_b^2}\Big)\right).
\tag{15}
\end{align*}
We can see that this separates into an x-dependent and a constant part just as in the standard Gaussian exponent form we mentioned above.
Finally insert \((15)\) into \((8)\):

\begin{align*}
\mathcal N(x;a,\sigma_a^2)\,\mathcal N(x;b,\sigma_b^2)
&=
\frac{1}{2\pi\sqrt{\sigma_a^2\sigma_b^2}}\,
\exp\!\left(
-\frac{(x-\mu_{ab})^2}{2\sigma_{ab}^2}
+\frac{1}{2\sigma_{ab}^2}\bigl(\mu_{ab}^2 - \sigma_{ab}^2 \Big(\frac{a^2}{\sigma_a^2} + \frac{b^2}{\sigma_b^2}\Big)\bigr)
\right)
\\[0.4em]
&=
\frac{1}{2\pi\sqrt{\sigma_a^2\sigma_b^2}}\,
\exp\!\left(\frac{1}{2\sigma_{ab}^2}\bigl(\mu_{ab}^2 - \sigma_{ab}^2 \Big(\frac{a^2}{\sigma_a^2} + \frac{b^2}{\sigma_b^2}\Big)\bigr)\right)
\exp\!\left(-\frac{(x-\mu_{ab})^2}{2\sigma_{ab}^2}\right).
\tag{16}
\end{align*}
Now we multiply and divide by \(\sqrt{2\pi\sigma_{ab}^2}\) to normalize it

\begin{align*}
\mathcal N(x;a,\sigma_a^2)\,\mathcal N(x;b,\sigma_b^2)
&=
\underbrace{
\frac{1}{2\pi\sqrt{\sigma_a^2\sigma_b^2}}\,\sqrt{2\pi\sigma_{ab}^2}
\exp\!\left(\frac{1}{2\sigma_{ab}^2}\bigl(\mu_{ab}^2 - \sigma_{ab}^2 \Big(\frac{a^2}{\sigma_a^2} + \frac{b^2}{\sigma_b^2}\Big)\bigr)\right)
}_{=:Z}
\\
&\qquad\qquad\times
\underbrace{
\frac{1}{\sqrt{2\pi\sigma_{ab}^2}}
\exp\!\left(-\frac{(x-\mu_{ab})^2}{2\sigma_{ab}^2}\right)
}_{=\mathcal N(x;\mu_{ab},\sigma_{ab}^2)}\\
&=\underline{Z \,\mathcal N(x;\mu_{ab},\sigma_{ab}^2).}
\tag{17}
\end{align*}
Now to get the final term $\mathcal{N}(x; a, \sigma_a^2)\,\mathcal{N}(x; b, \sigma_b^2) = \mathcal{N}(a; b, \sigma_a^2 + \sigma_b^2)\,\mathcal{N}(x; \mu_{ab}, \sigma_{ab}^2)$
with $\mu_{ab}, \sigma_{ab}^2$ we first simplify the constant exponent term in \(Z\).
We already know \(\sigma_{ab}^2\) from \((6)\).  
Now simplify the exponent
\(\dfrac{1}{2\sigma_{ab}^2}(\mu_{ab}^2-\sigma_{ab}^2 \Big(\frac{a^2}{\sigma_a^2} + \frac{b^2}{\sigma_b^2}\Big))\) using \((9)\) and getting rid of the $\mu_{ab}$ term

\begin{align*}
\mu_{ab}^2 - \sigma_{ab}^2 \Big(\frac{a^2}{\sigma_a^2} + \frac{b^2}{\sigma_b^2}\Big)
&= \sigma_{ab}^4 \Big(\frac{a}{\sigma_a^2} + \frac{b}{\sigma_b^2}\Big)^2 - \sigma_{ab}^2 \Big(\frac{a^2}{\sigma_a^2} + \frac{b^2}{\sigma_b^2}\Big)
\\[0.4em]
\Rightarrow\quad
\frac{1}{2\sigma_{ab}^2}\bigl(\mu_{ab}^2 - \sigma_{ab}^2 \Big(\frac{a^2}{\sigma_a^2} + \frac{b^2}{\sigma_b^2}\Big)\bigr)
&=
\frac{1}{2\sigma_{ab}^2}\bigl(\sigma_{ab}^4 \Big(\frac{a}{\sigma_a^2} + \frac{b}{\sigma_b^2}\Big)^2 - \sigma_{ab}^2 \Big(\frac{a^2}{\sigma_a^2} + \frac{b^2}{\sigma_b^2}\Big)\bigr)
\\[0.4em]
&=
\frac{\sigma_{ab}^2 \Big(\frac{a}{\sigma_a^2} + \frac{b}{\sigma_b^2}\Big)^2}{2} - \frac{\Big(\frac{a^2}{\sigma_a^2} + \frac{b^2}{\sigma_b^2}\Big)}{2}.
\tag{18}
\end{align*}
Now we plug in \((6)\) for $\sigma_{ab}^2$:

\begin{align*}
\sigma_{ab}^2 \Big(\frac{a}{\sigma_a^2} + \frac{b}{\sigma_b^2}\Big)^2
&=
\frac{\sigma_a^2\sigma_b^2}{\sigma_a^2+\sigma_b^2}
\left(\frac{a}{\sigma_a^2} + \frac{b}{\sigma_b^2}\right)^2
\\[0.4em]
&=
\frac{\sigma_a^2\sigma_b^2}{\sigma_a^2+\sigma_b^2}
\left(\frac{a\sigma_b^2 + b\sigma_a^2}{\sigma_a^2\sigma_b^2}\right)^2
\\[0.4em]
&=
\frac{\sigma_a^2\sigma_b^2}{\sigma_a^2+\sigma_b^2}
\cdot
\frac{(a\sigma_b^2 + b\sigma_a^2)^2}{\sigma_a^4\sigma_b^4}
\\[0.4em]
&=
\frac{(a\sigma_b^2 + b\sigma_a^2)^2}{(\sigma_a^2+\sigma_b^2)\sigma_a^2\sigma_b^2}.
\tag{19}
\end{align*}
If we insert it back into \((18)\) we have

\begin{align*}
\frac{\frac{(a\sigma_b^2 + b\sigma_a^2)^2}{(\sigma_a^2+\sigma_b^2)\sigma_a^2\sigma_b^2}}{2}
- \frac{\Big(\frac{a^2}{\sigma_a^2} + \frac{b^2}{\sigma_b^2}\Big)}{2}
&= \frac{1}{2}\left[
\frac{(a\sigma_b^2 + b\sigma_a^2)^2}{(\sigma_a^2+\sigma_b^2)\sigma_a^2\sigma_b^2}
- \left(\frac{a^2}{\sigma_a^2} + \frac{b^2}{\sigma_b^2}\right)
\right]
\tag{20}
\end{align*}
We put the second term over the common denominator \(\sigma_a^2\sigma_b^2\):

\[
\frac{a^2}{\sigma_a^2} + \frac{b^2}{\sigma_b^2}
= \frac{a^2\sigma_b^2 + b^2\sigma_a^2}{\sigma_a^2\sigma_b^2}.
\tag{21}
\]
and recieve

\begin{align*}
&\frac{1}{2}\left[
\frac{(a\sigma_b^2 + b\sigma_a^2)^2}{(\sigma_a^2+\sigma_b^2)\sigma_a^2\sigma_b^2}
- \frac{a^2\sigma_b^2 + b^2\sigma_a^2}{\sigma_a^2\sigma_b^2}
\right]
\\[0.4em]
&= \frac{1}{2}\left[
\frac{(a\sigma_b^2 + b\sigma_a^2)^2}{(\sigma_a^2+\sigma_b^2)\sigma_a^2\sigma_b^2}
- \frac{(a^2\sigma_b^2 + b^2\sigma_a^2)(\sigma_a^2+\sigma_b^2)}{(\sigma_a^2+\sigma_b^2)\sigma_a^2\sigma_b^2}
\right]
\\[0.4em]
&= \frac{1}{2}\,
\frac{(a\sigma_b^2 + b\sigma_a^2)^2
- (a^2\sigma_b^2 + b^2\sigma_a^2)(\sigma_a^2+\sigma_b^2)}
{(\sigma_a^2+\sigma_b^2)\sigma_a^2\sigma_b^2}.
\tag{22}
\end{align*}
Now we have to simplify the numerator as this is just a binomial formula and expand
\[
(a\sigma_b^2 + b\sigma_a^2)^2
= a^2\sigma_b^4 + 2ab\sigma_a^2\sigma_b^2 + b^2\sigma_a^4
\tag{23}
\]
and
\begin{align*}
(a^2\sigma_b^2 + b^2\sigma_a^2)(\sigma_a^2+\sigma_b^2)
&= a^2\sigma_b^2\sigma_a^2 + a^2\sigma_b^4
 + b^2\sigma_a^4 + b^2\sigma_a^2\sigma_b^2.
\tag{24}
\end{align*}
So the the numerator becomes

\begin{align*}
&\quad (a\sigma_b^2 + b\sigma_a^2)^2
      - (a^2\sigma_b^2 + b^2\sigma_a^2)(\sigma_a^2+\sigma_b^2)
\\[0.4em]
&= \bigl(a^2\sigma_b^4 + 2ab\sigma_a^2\sigma_b^2 + b^2\sigma_a^4\bigr)
   - \bigl(a^2\sigma_b^2\sigma_a^2 + a^2\sigma_b^4
           + b^2\sigma_a^4 + b^2\sigma_a^2\sigma_b^2\bigr)
\\[0.4em]
&= a^2\sigma_b^4 + 2ab\sigma_a^2\sigma_b^2 + b^2\sigma_a^4
   -a^2\sigma_b^2\sigma_a^2 - a^2\sigma_b^4
   -b^2\sigma_a^4 - b^2\sigma_a^2\sigma_b^2
\\[0.4em]
&= 2ab\sigma_a^2\sigma_b^2 - a^2\sigma_a^2\sigma_b^2 - b^2\sigma_a^2\sigma_b^2
\\[0.4em]
&= \sigma_a^2\sigma_b^2\,(2ab - a^2 - b^2)
\\[0.4em]
&= -\sigma_a^2\sigma_b^2\,(a^2 - 2ab + b^2)
\\[0.4em]
&= -\sigma_a^2\sigma_b^2\,(a-b)^2.
\tag{25}
\end{align*}
Now we can plug \((25)\) back into \((22)\), and finally obtain the standard gaussian exponent form of Z

\begin{align*}
&\frac{1}{2}\,
\frac{-\cancel{\sigma_a^2\sigma_b^2}\,(a-b)^2}
     {(\sigma_a^2+\sigma_b^2)\cancel{\sigma_a^2\sigma_b^2}}
\\[0.4em]
&= -\frac{(a-b)^2}{2(\sigma_a^2+\sigma_b^2)}.
\tag{26}
\end{align*}
Now since we have the exponent already only the normalization factor in \(Z\) needs to brought to standard form, which we can simplify again with replacing $\sigma_{ab}^2$, as follows

\begin{align*}
\frac{1}{2\pi\sqrt{\sigma_a^2\sigma_b^2}}\sqrt{2\pi\sigma_{ab}^2}
&=
\frac{1}{2\pi\sqrt{\sigma_a^2\sigma_b^2}}\,
\sqrt{2\pi\frac{\sigma_a^2\sigma_b^2}{\sigma_a^2+\sigma_b^2}}
\\[0.4em]
&=
\frac{1}{2\pi\sqrt{\cancel{\sigma_a^2\sigma_b^2}}}\,
\frac{\sqrt{2\pi}\sqrt{\cancel{\sigma_a^2\sigma_b^2}}}{\sqrt{\sigma_a^2+\sigma_b^2}}
\\[0.4em]
&=
\frac{1}{2\pi}\,
\frac{\sqrt{2\pi}\sqrt{1}}{\sqrt{\sigma_a^2+\sigma_b^2}}
\\[0.4em]
&=
\frac{1}{2\pi}\,
\frac{\sqrt{2\pi}\,1}{\sqrt{\sigma_a^2+\sigma_b^2}}
\\[0.4em]
&=
\frac{1}{2\pi}\,
\frac{\sqrt{2\pi}}{\sqrt{\sigma_a^2+\sigma_b^2}}
\\[0.4em]
&=
\frac{\sqrt{2\pi}}{2\pi\,\sqrt{\sigma_a^2+\sigma_b^2}}
\\[0.4em]
&=
\frac{\cancel{\sqrt{2\pi}}}{(\sqrt{2\pi})\cancel{^2}\,\sqrt{\sigma_a^2+\sigma_b^2}}
\\[0.4em]
&=
\frac{1}{\sqrt{2\pi}\,\sqrt{\sigma_a^2+\sigma_b^2}}
\\[0.4em]
&=
\frac{1}{\sqrt{2\pi(\sigma_a^2+\sigma_b^2)}}.
\tag{27}
\end{align*}
Combining \((26)\) and \((27)\) into Z, we obtain

\[
Z
=
\frac{1}{\sqrt{2\pi(\sigma_a^2+\sigma_b^2)}}\,
\exp\!\left(-\frac{(a-b)^2}{2(\sigma_a^2+\sigma_b^2)}\right).
\tag{28}
\]
which is exactly the Gaussian density

\[
Z = \mathcal N\bigl(a; b, \sigma_a^2+\sigma_b^2\bigr).
\tag{29}
\]
and so we obtained the final result where we exactly recovered the equation

\[
\mathcal N(x;a,\sigma_a^2)\,\mathcal N(x;b,\sigma_b^2)
=
\mathcal N\bigl(a; b, \sigma_a^2+\sigma_b^2\bigr)\,
\mathcal N\bigl(x;\mu_{ab},\sigma_{ab}^2\bigr)
\]

\newpage

\subsection*{Task 3.2 [2 points]}
Let \(p : \mathbb{R} \to \mathbb{R}\) denote the probability density function of a continuous random variable
\(X\), given as
\[
p(x) =
\begin{cases}
\dfrac{1}{2}x & 0 < x < 2\\[4pt]
0 & \text{otherwise}
\end{cases}
\]

\begin{enumerate}
    \item The corresponding \emph{cumulative distribution function} (CDF) is defined as
    \[
    F(x) = \int_{-\infty}^{x} p(z)\,dz.
    \]
    Write down \(F(x)\) as a simple function of \(x\) that does not involve an integral (i.e., solve the definite
    integral).

    \item Compute \(\mathbb P_X([{-0.5}, 0.5] \cup [1.5, 2])\) using only the CDF \(F\) and the properties of a probability measure.

    \item Compute \(\mathbb P_X(\{1\})\). Is it the same as \(p(1)\)?

    \item Prove or refute: \(\mathbb P_X([0, 1]) = \mathbb P_X([0, 1))\).

    \item Analytically compute \(\mathbb E_X[X]\).

    \item We can sample from \(p(x)\) using the \emph{inverse transform sampling} trick: First, sample
    \(u \sim \mathrm{Unif}([0, 1])\) and then compute \(x = F^{-1}(u)\) where \(F^{-1}\) denotes the inverse of \(F\).
    The result \(x\) is a proper sample from \(p\). Write down \(F^{-1}(u)\) and implement this sampling
    procedure in \texttt{monte\_carlo.py} (function \texttt{F\_inv}).

    \item Use this sampling procedure to estimate \(\mathbb E_X[X]\) via Monte Carlo: For all
    \(N \in \{100, 200, 300, \ldots, 10000\}\), compute the sample mean
    $\hat{\mathbb E}_N[X] := \frac{1}{N}\sum_{i=1}^{N} x_i$
    where \(x_i\) are i.i.d.\ samples from \(p\). Plot the sample mean as a function of \(N\) (i.e., \(N\) is shown
    on the x-axis, and the corresponding sample mean on the y-axis). Draw a horizontal line at the true
    expectation. Include this plot in your report.
\end{enumerate}


\newpage

\section*{3.2}

The density is
\[
p(x)=
\begin{cases}
\dfrac{1}{2}x & 0<x<2\\[3pt]
0 & \text{otherwise}
\end{cases}
\]

\subsection*{1.}
CDF $F(x)$.
Since we know that $x$ lives in [0,2] we can directly conclude that $x\le0$ = 0.

\[
F(x)=\int_{-\infty}^x p(z)\,dz=
\int_{0}^{x}\frac{z}{2}\,dz= \frac{x^{2}}{4}
\]
\[
F(x)=
\begin{cases}
0 & x\le 0,\\[4pt]
\displaystyle \frac{x^{2}}{4} & 0<x<2,\\[8pt]
1 & x\ge 2.
\end{cases}
\]

\subsection*{2.}
Compute $\mathbb P_X([{-0.5},0.5]\cup[1.5,2])$ using $F$.
Since $[{-0.5},0.5]$ and $[1.5,2]$ are disjoint we have
\[
\mathbb P_X([{-0.5},0.5]\cup[1.5,2])
= \big(F(0.5)-F(-0.5)\big)+\big(F(2)-F(1.5)\big).
\]
With using our defined $F$ from above
\[
F(-0.5)=0
\]
\[
F(0.5)=\frac{0.5^2}{4}=\frac{1}{16}
\]
\[
F(1.5)=\frac{1.5^2}{4}=\frac{9}{16}
\] 
\[
F(2)=1
\]
So we have

\begin{align*}
\mathbb P_X([{-0.5},0.5]\cup[1.5,2])&=\frac{1}{16}+\left(1-\frac{9}{16}\right)\\
&=\frac{1}{16}+\frac{7}{16}\\
&=\frac{8}{16}\\
&=\frac{1}{2}
\end{align*}

\subsection*{3.}
Compute $\mathbb P_X(\{1\})$ and compare to $p(1)$.\\
For a pdf with continuous random variables we know that just a single point is always 0 because it has no width and probability density $\neq$ density!
\[
\mathbb P_X(\{1\}) = 0.
\]
But $p(1)=\tfrac12$ and that's why $\mathbb P_X(\{1\})\neq p(1)$.

\subsection*{4.}
Prove or refute: $\mathbb P_X([0,1])=\mathbb P_X([0,1))$.\\
We have $[0,1]=[0,1)\,\cup\,\{1\}$, so
\[
\mathbb P_X([0,1])=\mathbb P_X([0,1))+\mathbb P_X(\{1\}).
\]
From 3. where we have said that $\mathbb P_X(\{1\})=0$, so it follows that $\mathbb P_X([0,1])=\mathbb P_X([0,1))$ is true.

\subsection*{5.}
Compute $\mathbb E_X[X]$.

\begin{align*}
\mathbb E[X]=\int_{x} p(x)\,x\,dx
&=\int_{0}^{2} \frac{x}{2}\cdot x\,dx\\
&=\frac12\int_{0}^{2}x^{2}\,dx\\
&=\frac12\cdot\Big(\frac{2^{3}}{3}-0\Big)\\
&=\frac12\cdot\frac{2^{3}}{3}\\
&=\frac12\cdot\frac{8}{3}\\
&=\frac{8}{6}\\
&=\frac{4}{3}
\end{align*}
So the expectation $\mathbb E_X[X]$ is $\frac{4}{3}$.

\subsection*{6.}

We are given the cumulative distribution function (CDF) of \(X\):
\[
F(x) =
\begin{cases}
0, & x \le 0, \\[4pt]
\dfrac{x^2}{4}, & 0 < x < 2, \\[6pt]
1, & x \ge 2.
\end{cases}
\]

\(F(x)\) increases continuously only on the interval \(0 < x < 2\) and on this interval we have
\[
u = \frac{x^2}{4}.
\]
and \(u\) ranges from \(0\) to \(1\) in that interval because
\[
F(0) = 0, \quad F(2) = 1 \quad \Rightarrow \quad 0 < u < 1.
\]
So to get the inverse we need to get x on one side

\begin{align*}
u = \frac{x^2}{4}\\
4u = x^2\\
4u = x^2.\\
x = 2\sqrt{u}
\end{align*}
Including the intervals for each case for completeness, we now have the inverse function defined as

\[
F^{-1}(u) = x =
\begin{cases}
0, & u = 0, \\[4pt]
2\sqrt{u}, & 0 < u < 1, \\[4pt]
2, & u = 1.
\end{cases}
\]

\subsection*{7.}

\begin{figure}[H]
\centering
\includegraphics{MC_Estimate.png}
\end{figure}

\end{document}
